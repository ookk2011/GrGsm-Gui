%###########################################################################
%
%   Kapitel 3
%
%###########################################################################

\chapter{Hardwareaufbau des Handhelds}

Im Folgenden wird erklärt, wie die verbauten Komponenten zusammenzufügen sind, sollte der Bedarf bestehen das Handheld erneut aufzubauen, etwas zu modifizieren, oder sich ein Defekt einschleicht. Die Anleitung für das Case stammt von \url{Adafruit.com} und wird als PDF Datei der DVD beiliegen. Die Bilder welche im folgenden Abschnitt verwendet werden stammen ebenfalls aus dieser Anleitung.

\section{Druck des Cases}

Die CAD Dateien zum Druck des Cases sind auf der DVD hinterlegt und können dafür verwendet werden das Case erneut drucken zu lassen, beziehungsweise, falls notwendig, zu modifizieren. 

Das empfohlene Material ist PLA wobei ABS, Nylon, copperfill, bamboofill, oder PET ebenfalls verwendet werden können. Da bei Herr Strohrmann HiWis nur ABS auf Lager war, wurde dieses verwendet. 

Über den Sommer haben wir die Erfahrung gemacht, dass sich das Material stark verzieht wenn es Wärme ausgesetzt wird. Sollte dies wieder vorkommen, 	 kann ein vorsichtig verwendeter Föhn Abhilfe verschaffen, um das Case wieder in Form zu bringen. 

\section{Benötigten Teile}
Folgende Teile werden benötigt, um das Handheld aufzubauen.

\newpage
\subsection{Hardware}
\begin{itemize}
\item Pi Foundation PiTFT - 7" Touchscreen Display 
\item Raspberry Pi 3 
\item 200mm Flex Displaykabel
\item Adafruit PowerBoost 1000C
\item 2500mAh LiPo Akku
\item SPDT Schalter
\item 16GB  Micro SD Karte (r: 95MB/s, w: 60MB/s)
\end{itemize} 

\subsection{Werkzeuge und Ergänzendes}
Zudem braucht man noch gewisses Werkzeug:
\begin{itemize}
\item 3D Drucker 
\item Filament  
\item Kreuzschlitzschraubenzieher 
\item Lot 
\item Litzen mit 1,5 mm$^2$
\item Kabelbinder
\item M3 x .5 x 6M Schrauben x12
\end{itemize}

\section{Zusammenbau}

Der Zusammenbau des Tablets ist selbsterklärend.

\mywidthpicture{zusammenbau}{Zusammenbau der Caseteile}{fig:case-fig}{0.0}{\textwidth}

Nachdem die gedruckten Teile wie in Abbildung~\ref{fig:case-fig} zusammengebaut wurden, müssen nur noch die einzelnen Komponenten an ihren Platz geschraubt werden, wie in Abbildung~\ref{fig:case2-fig} zu sehen ist.

\mywidthpicture{zusammenbau2}{Einbau der Elektronik}{fig:case2-fig}{0.0}{\textwidth}

Anschließend noch den Akku, mit einem Kabelbinder, an seinem Rahmen befestigen und dann über dem Displaytreiber festschrauben: siehe Abbildung~\ref{fig:case3-fig}

\mywidthpicture{zusammenbau3}{Einsetzen des Akkus}{fig:case3-fig}{0.0}{\textwidth}

Zu guter Letzt noch das Flachbandkabel des Displays in den Raspberry stecken. Hierfür zuerst die graue Lasche an beiden Seiten nach oben ziehen, das Kabel bis zum Anschlag einlegen und die Lasche gleichmäßig wieder eindrücken. Zu guter Letzt noch den Deckel aufsetzen, verschrauben und das Handheld ist fertig.


\subsection{Verkabelung}
\mywidthpicture{Verkabelung}{Verkabelung im Inneren des Case}{fig:Verkabelung-fig}{0.0}{\textwidth}

Die Pole EN und GND des Adafruit PowerBoost1000C werden an den Schalter herausgeleitet. GND wird hierbei mit dem mittleren Pin des Schalters verbunden. 

Der Akku wird über den JST Stecker an die PowerBoost1000C angeschlossen. 

Der positive Ausgang des PowerBoost1000C wird mit dem GPIO \# 2 und der Negative mit dem GPIO \# 6 verbunden. 

Der 5V Pin des Displaytreibers wird mit den GPIO \# 4 und GND an GPIO \# 9 des Raspberrys verbunden. 
\mywidthpicture{pinout}{Pinbelegung des Raspberry Quelle:microsoft.com}{fig:pinout-fig}{0.0}{0.9\textwidth}

\section{Stromversorgung}

Herzstück der Stromversorgung ist ein Adafruit PowerBoost1000C. Hierbei handelt es sich um eine Elektronik wie sie in vielen Powerbanks zu finden ist. Das heißt einerseits hat man einen MikroUSB Eingang, über den der LiPo Akku geladen werden kann, andererseits gibt es einen USB Ausgang an dem die 3,7V des Akkus auf 5V hochgeregelt ausgegeben werden. Für das Laden des Akkus ist der MCP73871 von Microchip verantwortlich, der DC/DC Boostconverter ist von Texas Instruments: TPS6109. Der USB Connector wurde nicht verbaut und die 5V werden über die verlöteten Litzen direkt auf den Raspberry Pi geleitet. 

Üblicherweise können 1A oder Spitzenwerte bis zu 2,5A, aus dem Akku gezogen werden. Die Maximale Stromaufnahme des Raspberry beträgt 2,5A.  
