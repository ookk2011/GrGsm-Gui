%###########################################################################
%
%   Zusammenfassung und Ausblick
%
%###########################################################################
\chapter{Zusammenfassung}

In dieser Projektarbeit ist es uns gelungen ein Handheld zum Auslesen des BCCH zu konzipieren, diesen aufzubauen und in Betrieb zu nehmen. Als Herzstück dient uns ein von Kali Linux betriebener Raspberry Pi 3, das dazugehörige Display mit Toucheingabe und ein DVB-T-Stick. Letzterer wurde nach dem Prinzip des Software Defined Radio verwendet, welches besagt, dass möglichst viel der Signalverarbeitung in der Software durchgeführt wird. So können die analogen Komponenten der Kommunikationssysteme einfach gehalten werden.

Mit den GNU Radio Blöcken "gr-gsm" von Piotr Krysik ist es uns gelungen den Informationsstring des BCCH auszulesen und mit einer eigens geschriebenen \ac{GUI} darzustellen. Die Darstellung der \ac{GUI} und der Suchvorgang der \ac{GSM} Basisstationen laufen parallel auf zwei Threads um eine laufende Aktualisierung zu ermöglichen.

Mit dem fertigen Handheld ist eine mobile Nutzung von bis zu 5 Stunden möglich. So kann auch die Netzabdeckung der entferntesten Orte analysiert werden.