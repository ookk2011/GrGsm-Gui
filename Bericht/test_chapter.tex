%###########################################################################
%
%   Kapitel 1
%
%###########################################################################
\chapter{Einige Features}
Hier werden einige Features gezeigt.

\section{Ein Unterabschnitt}
Die Tabelle~\ref{tab:example-table} ist ein Beispiel für eine wissenschaftliche Tabelle (ohne vertikale Trennlinien).

\begin{table}[ht]
\caption{Beispieltabelle}
\centering
\begin{tabular}{@{}ccc@{}} \toprule
    Apfel & Birne & Katze \\
    \SI{200}{g} & \SI{180}{g} & \SI{3.4}{kg} \\ \bottomrule
\end{tabular}
\label{tab:example-table}
\end{table}

So zitiert man \cite{slam06}. Man kann auch mehrere Quellen auf einmal referieren \cite{pomax16, dlib09, biagiotti_open-loop_2012, hunter_matplotlib:_2007}

In Abbildung~\ref{fig:example-fig} ist eine Kurve dargestellt.
\mywidthpicture{fail_path_linearization}{Beispielbild}{fig:example-fig}{0.0}{\textwidth}

Inkscape erlaubt es eps-Files und den Text getrennt zu exportieren, auf diese Weise kann man die Schrift nachträglich anpassen. Ein Beispiel dazu ist in Abbildung~\ref{fig:inkscape-fig} dargestellt.
\myinkscapepicture{mr_kinmatic_model.eps_tex}{Inkskape Bild mit Text}{fig:inkscape-fig}{0.6\textwidth}

Außerdem gibt es eine Umgebung zum Schreiben von Pseudo-Code. Ein Beispiel dazu ist in Algorithmus~\ref{alg:example-alg} dargestellt.
\begin{algorithm}
\begin{algorithmic}[1]
\Procedure{AddTwoNumbers}{$x$, $y$}
	\State $sum \gets x + y$
   	\If{$sum = 42$}
		\State \Call{FixEverything}{}
   	\EndIf
\EndProcedure
\end{algorithmic}
\caption{Beispielalgorithmus}
\label{alg:example-alg}
\end{algorithm}

Das Paket acronym handelt Abkürzungen automatisch. Bei der ersten Verwendung sieht das so aus: \ac{UML}. Wenn die Abkürzung nochmal verwendet wird, steht da nur noch \ac{UML}.


